\chapter*{Preface}
%\preface

In doing the data analysis in my own research work, I was often slowed down by two things: 1) I did not know enough statistics, and 2) the books available would provide a theoretical background, but no real practical help. The book you are holding in your hands (or on your tablet or laptop) is intended to be the book that would have solved just this problem. It should provide enough basic understanding so you know what you are doing; and it should provide you with the tools to do so. In providing statistical solutions for the most basic statistical problem, I believe that I cover at least 90\% of the problems that most physicists, biologists, and medical doctors encounter in their work. So if you are the typical graduate student working on your degree, chances are that you will find the solution - explanation and source-code - here.

In contrast, for serious statistical analyses statistical modeling is state-of-the-art. But for most medical research and life science applications, hypothesis tests form the common ground for the statistical analysis. This is the reason I have focussed on statistical basics and hypothesis tests, and give only a brief outlook at other statistical approaches. I am well aware that most of the tests presented in this book can also be done with the methods of statistical modeling. But in many cases, this is not the jargon used in typical life science journals. Advances statistical analysis goes beyond the scope of this book, and - to be frank - beyond my statistical knowledge.

My motivation to provide the solutions in Python are based on two considerations. One is that I would like them to be available to everyone. While commercial solutions like \emph{Matlab}, \emph{SPSS}, \emph{Minitab} etc. are available, most of us can only use them legally as long as they are in academia. In contrast, \emph{Python} is completely free, as in "free beer". The second reason is that \emph{Python} is the most beautiful coding language that I have yet encountered; and around 2010 \emph{Python} and its documentation had matured to
the point where one could use it without being a hacker. All together, this book and Python provides a free, beautiful package that covers all the statistics that most researchers need to do in their lifetime. OK, for really serious statistical modeling :math:`R` still sets the standard.
But most will be more than happy with the tools that the \emph{Python} ecosystem offers today.

\section*{For whom this book is}

This book assumes that

\begin{itemize}
  \item you have some basic programming experience (If you have zero prior programming experience, you may want to start out with getting going with \emph{Python}, using some of the great links given in the text. Starting programming \emph{and} starting statistics may be a bit much at a time.)
  \item you have some data that you want to analyze (For almost all cases, a working \emph{Python} program is provided. All you have to do is select the right program, adjust it so that it reads in your data, and interpret the results.)
  \item you are familiar with the basic ideas of statistics, but are not a statistics expert (If you are already a statistics expert, the online help in \emph{Python} will be sufficient to allow you to do most of your data analysis right away.)
\end{itemize}


The idea of this book is to give you all (or at least most of) the tools that you will need for your statistical data analysis. Thereby I try to provide the background required to understand what you are doing. I will not proof any theorems, and won't indulge in mathematics where it is unnecessary. This approach explains why so much code is included: in principle, you have to define our problem, select the corresponding program, and adapt it to your needs. This should allow you to get going quickly, even if you have little Python experience. This is also the reason why I have not provided the software as a Python module, since I expect that you have to tailor each program to your specific setup (data format, etc).

\section*{How to use this book}

\begin{itemize}
  \item If you just want to look something up, simply go to the HTML-version of the book \url{http://work.thaslwanter.at/Stats/html}.

  \item If you want to go through it systematically, or if you prefer to read printed material, you may want to download the \href{http://work.thaslwanter.at/Stats/StatsIntro.pdf}.

  \item If you want to get the whole package, and/or if you want to contribute to the book, clone the PDF-version of the book \url{https://github.com/thomas-haslwanter/statsintro} {github repository of the book}, which  includes all the Python programs, the sample data used in the book, the TEX-files, RST-files, and all the images.

   If you have never used \emph{github}, you might want to check out this introduction to  \emph{github} \url{https://help.github.com/articles/set-up-git}. But don't be  scared off, you can download individual files easily from your
   web-browser.

\end{itemize}

This book is organized into three parts
\begin{description}
  \item[Part 1] gives an introduction \emph{Python}: how to set it up, simple programs to get started, and tips how to avoid some common mistakes. (Chapters 1 \& 2)
  \item[Part 2] provides an introduction to statistical analysis. How to visualize data, design a study, start the data analysis; probability distributions; and an overview of the most important hypothesis tests in life sciences. For each test a \emph{Python} program is provided, showing how the tests can be implemented.
  \item[Part 3] provide an outlook to advanced statistical analysis procedures, with an introduction to statistical modeling, and a presentation of the basic ideas of Bayesian Statistics. I have also included tests on discrete data in this section, such as logistic regression, as it utilized "generalized linear models" which I regard as an advanced topic.
\end{description}

Code samples are marked as follows

    \raisebox{-0.35\height}{\includegraphics[scale=0.25]{../Images/python.jpg}} Python code samples, listed in the Appendix.

\section*{Contributor List}

If you have a suggestion or correction, please send email to
thomas.haslwanter@fh-linz.at. If I make a change based on your feedback,
I will add you to the contributor list (unless you ask to be omitted).

If you include at least part of the sentence the error appears in, that
makes it easy for me to search. Page and section numbers are fine, too,
but not as easy to work with. Thanks!

\begin{itemize}
  \item Connor Johnson wrote a very nice blog explaining the results of
  statsmodels OLS command, which formed the basis of a large part of the
  section on \emph{Statistical Models}.

  \item To demonstrate Bayesian statistics and MCMC-models, I took the example of the Challenger disaster from the excellent open source e-book Probabilistic-Programming-and-Bayesian-Methods-for-Hackers by Cam Davidson Pilon.

  \item Fabian Pedregosa's blog on ordinal logistic regression allowed me to include a topic that is admittedly beyond my own skills.
\end{itemize}

